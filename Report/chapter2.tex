\chapter{روش‌های مبتنی بر ساختار}

روش‌های خلاصه‌سازی مبتنی بر ساختار طیفی از رویکردها را در بر می‌گیرند که از ویژگی‌های ساختاری متن ورودی برای تولید خلاصه‌های مختصر و منسجم استفاده می‌کنند. در این رویکرد اطلاعات مهم متن به یک ساختار از پیش تعریف شده داده می‌شود و خلاصه با توجه به ساختار ایجاد می‌شود. در این فصل روش‌های مبتنی بر درخت 
\LTRfootnote{tree-based}
، مبتنی بر قالب
\LTRfootnote{template-based}
، مبتنی بر هستان‌شناسی
\LTRfootnote{ontology-based}
، عبارت مقدمه و بدنه
\LTRfootnote{lead-and-body phrase}
، مبتنی بر گراف 
\LTRfootnote{graph-based }
و مبتنی بر قانون
\LTRfootnote{rule-based}
 مورد بررسی قرار می‌گیرد. 
 
\section{روش مبتنی بر درخت}
روش مبتنی بر درخت در خلاصه‌سازی متن شامل استفاده از درخت‌های وابستگی برای نمایش سند متنی است. متن مبدأ ابتدا به درخت‌های وابستگی تبدیل می‌شود، که سپس در یک درخت واحد ادغام می‌شوند. سپس این درخت وابستگی ادغام شده به جمله ای تبدیل می شود که به عنوان جمله ترکیبی شناخته می شود. فرآیند تبدیل درخت وابستگی به رشته ای از کلمات را خطی سازی درخت می گویند. عملکرد این روش به انتخاب تجزیه کننده و حفظ وابستگی بین کلمات بستگی دارد. تکنیک‌های مختلفی پیشنهاد شده‌اند، مانند استفاده از تجزیه‌کننده‌های کم عمق برای ترکیب جملات مشابه، حذف زیردرخت‌های درخت‌های وابستگی برای فشرده‌سازی، و تولید درخت‌های تودرتو با استفاده از ساختارهای بلاغی و تجزیه وابستگی. به طور کلی، روش مبتنی بر درخت با استفاده از ساختار سند متنی، با هدف ایجاد خلاصه‌های مختصر و منسجم است.
این روش به عملکرد  تجزیه کننده‌ها وابسطه است و این باعث محدود شدن کارایی می‌شود\cite{andhale2016overview}.

\section{روش مبتنی بر قالب}
روش های مبتنی بر الگو در خلاصه سازی متن شامل استفاده از قالب های از پیش تعریف شده برای نمایش سند است. این قالب ها برای مطابقت با الگوها و قوانین خاص در محتوای متنی طراحی شده اند و امکان استخراج اطلاعات مرتبط را فراهم می کنند که می توان آن ها را در فضای قالب ترسیم کرد. این فرآیند شامل تطبیق متن با این الگوها و قوانین برای شناسایی محتوای متناسب با الگو است، که سپس محتوای خلاصه را نشان می دهد. این روش بسیار منسجم است زیرا خلاصه هایی تولید می کند که به ساختار و قالب قالب ها پایبند هستند. با این حال، یکی از چالش های پیش روی روش های مبتنی بر الگو، نیاز به تجزیه و تحلیل معنایی دقیق است، چرا که قالب ها نیاز به محتوای خاص و مرتبط برای پر شدن دارند
\cite{andhale2016overview}.
\section{روش مبتنی بر هستان شناسی}
روش مبتنی بر هستی‌شناسی در خلاصه سازی متن شامل استفاده از پایگاه دانش یا هستی‌شناسی برای بهبود فرآیند خلاصه سازی است. این روش از این واقعیت بهره می‌برد که بسیاری از اسناد موجود در اینترنت به حوزه‌های خاصی با واژگان محدود مرتبط هستند که می‌توانند توسط هستی‌شناسی بهتر نمایش داده شوند. هستی‌شناسی نامگذاری و تعریف رسمی انواع موجودیت مربوط به یک دامنه خاص را ارائه می دهد که به عنوان پایگاه دانش عمل می کند. با استفاده از هستی‌شناسی، سیستم خلاصه سازی می تواند نمایش معنایی محتوای اطلاعات را بهبود بخشد و بسط پرس و جو را انجام دهد. تکنیک های مختلفی پیشنهاد شده است، مانند استفاده از هستی‌شناسی برای ساخت یک مدل معنایی، نگاشت جملات به گره های هستی‌شناسی، و محاسبه امتیاز مربوط به موجودیت برای رتبه بندی جملات. به طور کلی، روش مبتنی بر هستی‌شناسی از دانش خاص دامنه برای ایجاد خلاصه های دقیق تر و آموزنده تر استفاده می کند
\cite{andhale2016overview}.
لی و همکاران  یک سیستم فازی را ارائه کرد که از هستی‌شناسی طراحی شده توسط متخصص حوزه اخبار استفاده می کند. جملات بر اساس طبقه بندی کننده اصطلاحی که از هستی شناسی استفاده می کند، طبقه بندی می شوند. مکانیزم استنتاج فازی درجه عضویت برای هر جمله را با توجه به طبقه بندی کننده اصطلاح براساس هستی شناسی دامنه محاسبه می کند
\cite{lee2005fuzzy}.

\section{روش  عبارت مقدمه و بدنه}

روش عبارت مقدمه و بدنه یک رویکرد خلاصه‌سازی متن است که بر شناسایی و بازنگری جملات اصلی، معروف به جملات اصلی، در یک سند تمرکز دارد. این جملات اصلی معمولاً آموزنده هستند و خلاصه خوبی از محتوا ارائه می دهند. این روش شامل درج و جایگزینی عبارات در جمله اصلی برای ایجاد تجدید نظرهای معنایی مناسب است. با بازنویسی تکراری جمله اصلی، جملات خلاصه جدیدی تولید می شوند. با این حال، یکی از محدودیت های این روش این است که تجزیه می تواند عملکرد آن را کاهش دهد، و هیچ مدل تعمیم یافته ای برای خلاصه سازی وجود ندارد
\cite{andhale2016overview}.
ایشیکاوا و همکاران  روش خلاصه سازی ترکیبی مبتنی بر روش فرکانس عبارت
\LTRfootnote{Term frequency (TF) }
 و عبارت مقدمه و بدنه پیشنهاد کردند. تابع توزیع زاویه ای ضربدر بسامد عبارت که وزن را به هر جمله برای شناسایی اهمیت اختصاص می دهد. دستورها براساس اهمیت برای نوشتن خلاصه رتبه بندی می شوند
\cite{Ishikawa2001HybridTS}.

\section{روش مبتنی بر گراف}
روش مبتنی بر نمودار یک رویکرد خلاصه سازی است که هر جمله در یک سند را به عنوان یک راس در یک نمودار نشان می دهد. جملات بر اساس روابط معنایی با یال ها به هم متصل می شوند و وزن یال ها نشان دهنده قدرت رابطه است. سپس از یک الگوریتم رتبه بندی نمودار برای تعیین اهمیت هر جمله در نمودار استفاده می شود. جملات با اهمیت بالاتر، که با وزن های بالاتر یا ارتباط بیشتر نشان داده می شوند، مهم تر در نظر گرفته می شوند و در خلاصه گنجانده می شوند. این روش نیازی به دانش عمیق زبانی یا حوزه ای ندارد و می تواند با انتخاب جملاتی با اهمیت بالا، خلاصه های مختصر و منسجمی ایجاد کند
\cite{andhale2016overview}.
مالیروس و اسکیانیس  از مرکزیت گره برای نشان دادن اهمیت یک اصطلاح در سند استفاده می کنند. مرکزیت های گره محلی و جهانی برای وزن دهی عبارت در نظر گرفته می شوند تا خلاصه را شکل دهند
\cite{GraphBased}.
\section{روش مبتنی بر قانون }
در این تکنیک، اسنادی که باید خلاصه شوند از نظر طبقات و فهرست جنبه ها به تصویر کشیده می شوند. ماژول انتخاب محتوا، مؤثرترین نامزد را از میان مواردی که توسط قوانین استخراج داده ایجاد می شود، انتخاب می کند تا به یک یا بسیاری از جنبه های یک دسته پاسخ دهد. در نهایت، الگوهای تولید برای تولید جملات طرح کلی استفاده می شود
\cite{Moratanchsurvey}.







% !TeX spellcheck = fa
%% -!TEX root = AUTthesis.tex
% در این فایل، عنوان پایان‌نامه، مشخصات خود، متن تقدیمی‌، ستایش، سپاس‌گزاری و چکیده پایان‌نامه را به فارسی، وارد کنید.
% توجه داشته باشید که جدول حاوی مشخصات پروژه/پایان‌نامه/رساله و همچنین، مشخصات داخل آن، به طور خودکار، درج می‌شود.
%%%%%%%%%%%%%%%%%%%%%%%%%%%%%%%%%%%%
% دانشکده، آموزشکده و یا پژوهشکده  خود را وارد کنید
\faculty{دانشکده مهندسی کامپیوتر}
% گرایش و گروه آموزشی خود را وارد کنید
\department{}
% عنوان پایان‌نامه را وارد کنید
\fatitle{روش‌های خلاصه‌سازی انتزاعی }
% نام استاد(ان) راهنما را وارد کنید
\firstsupervisor{دکتر رضا صفابخش}
%\secondsupervisor{استاد راهنمای دوم}
% نام استاد(دان) مشاور را وارد کنید. چنانچه استاد مشاور ندارید، دستور پایین را غیرفعال کنید.
%\firstadvisor{نام کامل استاد مشاور}
%\secondadvisor{استاد مشاور دوم}
% نام نویسنده را وارد کنید
\name{زهرا }
% نام خانوادگی نویسنده را وارد کنید
\surname{زنجانی}
%%%%%%%%%%%%%%%%%%%%%%%%%%%%%%%%%%
\thesisdate{1402- 1401}

% چکیده پایان‌نامه را وارد کنید
\fa-abstract{
	خلاصه‌سازی نقش مهمی در علم اطلاعات و بازیابی دارد، زیرا ارتباط نزدیکی با فشرده‌سازی داده‌ها و درک اطلاعات دارد. توانایی تولید خلاصه‌های مناسب می‌تواند موجب بهبود کارآمدی سیستم‌های استخراج اطلاعات و صرفه جویی در وقت انسان‌ها شود. خلاصه‌سازی خودکار به عنوان یک کار برجسته در پردازش زبان طبیعی 
	\LTRfootnote{natural language processing (NLP)}
	ظاهر شده است. با این حال، علیرغم اهمیت آن، چالش‌های خلاصه‌سازی خودکار تا حد زیادی حل نشده باقی مانده است. این گزارش مروری جامع از وضعیت فعلی خلاصه‌سازی خودکار ارائه می‌کند و رویکردها، تکنیک ها و معیارهای ارزیابی مختلف به کار گرفته شده در این زمینه را بررسی می‌کند.
}


% کلمات کلیدی پایان‌نامه را وارد کنید
\keywords{ خلاصه‌سازی متن، پردازش زبان طبیعی، یادگیری عمیق،یادگیری تقویتی}



\AUTtitle
%%%%%%%%%%%%%%%%%%%%%%%%%%%%%%%%%%
\vspace*{7cm}
\thispagestyle{empty}
\begin{center}
\includegraphics[height=5cm,width=12cm]{besm}
\end{center}